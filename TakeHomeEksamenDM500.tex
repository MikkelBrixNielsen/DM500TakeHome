\documentclass[20pt]{article}
\usepackage{graphicx}
\usepackage{tabularx}
\usepackage[utf8]{inputenc}
\usepackage{amsmath}
\usepackage{amsfonts} 
\usepackage{amssymb}
\usepackage{enumitem}
\usepackage[danish]{babel}
\usepackage[utf8]{inputenc}
\usepackage{fancyhdr}
\usepackage{lastpage}
\pagestyle{fancy}
\fancyhf{}

\title{Take-Home Eksamen DM500 Studieintroduktion til datalogi, Efteråret 2021}
\author{Danny Nicolai Larsen (dalar21), Steffen Bach (stbac21),\\ Mikkel Brix Nielsen (mikke21) \\ Studiegruppe 11}
\date{\today}
\renewcommand{\headrulewidth}{0pt}
\chead{Git og Latex introduktion - DM500 Studieintroduktion}
\cfoot{\thepage\ of \pageref{LastPage}}

\begin{document}
	\clearpage
	\maketitle
	\thispagestyle{empty}
	\newpage
	\section*{Opgave 1 - Eksamen januar 2009 opgave 1}
	Dem første opgave der vil blive løst kommer fra Eksamens sættet fra januar 2009, og er opgave nr.2.
	I denne opgaven skal vi betragte de to matricer:
	$$ 
	A = 
	\begin{bmatrix} 
		1 & 1 & 2 \\
		0 & 1 & 0 \\
		1 & 0 & 3 \\
	\end{bmatrix}
 	\quad og \quad B = 
	\begin{bmatrix} 
		0 & 2 & 1 \\
		0 & 0 & 0 \\
		2 & 0 & 1 \\
	\end{bmatrix}
	\quad
	$$
	Til denne opgaver forekommer 2 delopgaver. I den ene delopgave skal matricen der forekommer ved \(A+B\) beregnes, og i den anden delopgave skal matricen, der forekommer ved \(A*B\) beregnes.	
	
	\subsection*{Opgave a)}
	Lad matricen der forekommer ved \(A+B\) være C, så kan værdien, der skal indsættes i  matrice C, på index (i, j), være summen af tallene, der findes i matrice A på index (i, j) samt værdien, der findes i matrice B på index (i, j).
	
	\begin{equation}
		C = A+B
	\end{equation}

	\begin{equation}
			C = 
		\begin{bmatrix} 
			1 & 1 & 2 \\
			0 & 1 & 0 \\
			1 & 0 & 3 \\
		\end{bmatrix}
		\quad + \quad
		\begin{bmatrix} 
			0 & 2 & 1 \\
			0 & 0 & 0 \\
			2 & 0 & 1 \\
		\end{bmatrix}
		\quad	
	\end{equation}
	\\ 
	\begin{equation}
		= 
		\begin{bmatrix} 
			1+0 & 1+2 & 2+1 \\
			0+0 & 1+0 & 0+0 \\
			1+1 & 0+0 & 3+1 \\
		\end{bmatrix}
		= 
		\begin{bmatrix} 
			1 & 3 & 3 \\
			0 & 1 & 0 \\
			2 & 0 & 4 \\
		\end{bmatrix}
	\end{equation}
	\\
	\begin{equation}
		C = 
		\begin{bmatrix} 
			1 & 3 & 3 \\
			0 & 1 & 0 \\
			2 & 0 & 4 \\
		\end{bmatrix}
	\end{equation}
	\\
	Og herved er delopgave a løst, og den matrice der forekommer ved at addere matrice A med matrice B, \(A+B\), vil svare til matrice C.
	\newpage
	\subsection*{Opgave b)}
	Lad matricen der forekommer ved \(A*B\) være C, så kan værdien, der skal indsættes i matricen C på index (i, j) være summen af produkterne, der optår, når index (i, j) fra matrice A multipliceres med index (j, i) fra matrice B, for alle elementer på række i fra matrice A og kolonne j fra matrice B.
		\begin{equation}
		C = A*B
	\end{equation}
	
	\begin{equation}
		C = 
		\begin{bmatrix} 
			1 & 1 & 2 \\
			0 & 1 & 0 \\
			1 & 0 & 3 \\
		\end{bmatrix}
		\quad * \quad
		\begin{bmatrix} 
			0 & 2 & 1 \\
			0 & 0 & 0 \\
			2 & 0 & 1 \\
		\end{bmatrix}
		\quad	
	\end{equation}
	\\ 
	\begin{equation}
		= 
		\begin{bmatrix} 
			1*0+1*0+2*2 & 1*2+1*0+2*0 & 1*1+1*0+2*1 \\
			0*0+1*0+0*2 & 0*2+1*0+0*0 & 0*1+1*0+0*1 \\
			1*0+0*0+3*2 & 1*2+0*0+3*0 & 1*1+0*0+3*1 \\
		\end{bmatrix}
		= 
		\begin{bmatrix} 
			4 & 2 & 3 \\
			0 & 0 & 0 \\
			6 & 2 & 4 \\
		\end{bmatrix}
	\end{equation}
	\\
	\begin{equation}
		C = 
		\begin{bmatrix} 
			4 & 2 & 3 \\
			0 & 0 & 0 \\
			6 & 2 & 4 \\
		\end{bmatrix}
	\end{equation}
	\\
	Og herved er delopgave b løst, og den matrice der forekommer ved at multiplicere matrice A med matrice B, \(A*B\), vil svare til matrice C.
	\newpage
		
	\section*{Opgave 2 - Reeksamen februar 2015 opgave 2}
	\begin{enumerate}[label=(\alph*)]
		\item 
		Hvilke af følgende udsagn er sande?
		\begin{align*}
		\forall x \in \mathbb{N}\!&:\exists y \in \mathbb{N}\!: x<y\\
		\forall x \in \mathbb{N}\!&:\exists !y \in \mathbb{N}\!: x<y\\
		\exists x \in \mathbb{N}\!&:\forall y \in \mathbb{N}\!: x<y
		\end{align*}
		Første udsagn siger at for alle \(x\), existerer der mindst et \(y\) hvorom det gælder at \(x<y\)
		\\
		Dette er sandt, da man kan sætte \(y=x+1\)
		\\
		\\
		Andet udsagn siger at for alle \(x\), eksisterer der \emph{kun} et \(y\) hvorom det gælder at \(x<y\)
		\\
		Dette er falsk da der findes flere tal der er større end \(x\) f.eks. \(x+1, x+2\) osv.
		\\
		\\
		Tredje udsagn siger at man kan vælge et enkelt \(x\), hvor ligemeget hvilket \(y\) man vælger gælder det at \(x<y\)
		\\
		Dette er falsk, da man ligemeget hvilket \(x\) der er valgt, vil man kunne vælge \(y-1\)
		\\
		\item
		Angiv negeringen af udsagn 1. fra spørgsmål a).\\
		Negerings-operatoren (\(\neg\)) må ikke indgå i dit udsagn.
		\begin{align*}
		\neg(\forall x \in \mathbb{N}\!&:\exists y \in \mathbb{N}\!: x<y) \Leftrightarrow\\
		\exists x \in \mathbb{N}\!&:\forall y \in \mathbb{N}\!: \neg(x<y)\Leftrightarrow\\
		\exists x \in \mathbb{N}\!&:\forall y \in \mathbb{N}\!: x \geq y
		\end{align*}
		Først negeres kvantorerne, og derefter negeres det logiske udsagn.
		\begin{center}
			$\underline{\underline{\exists x \in \mathbb{N}\!:\forall y \in \mathbb{N}\!:x \geq y}}$\\
		\end{center}
	\end{enumerate}
	\newpage
	\section*{Opgave 3 - Reeksamen februar 2015 opgave 3}
Lad \(R\), \(S\) og \(T\) være binære relationer på mængden \(\{1, 2, 3, 4\}\).
\begin{enumerate}[label=(\alph*)]
	\item 
	Lad \(R\) = \(\{(1, 1), (2, 1), (2, 2), (2, 4), (3, 1), (3, 3), (3, 4), (4, 1), (4, 4)\}\)\\
	Er \(R\) en partiel ordning?\\ \\
	Ja \(R\) er en partiel ordning da den er både refleksiv, transitiv og anti-symmetrisk\\
	\item
	Lad \(S\) = \{(1, 2), (2, 3), (2, 4), (4, 2)\}\\
	Angiv den transitive lukning af \(S\) \\ \\
	Den transitive lukning af \(S\) er : \\
	\(\{(1, 2), (1, 3), (1, 4), (2, 2), (2, 3), (2, 4), (4, 2), (4, 4)\}\)\\
	\item
	Lad \(T\) = \{(1, 1), (1, 3), (2, 2), (2, 4), (3, 1), (3, 3), (4, 2), (4, 4)\}\\
	Bemærk, at \(T\) er en ækvivalens-relation.\\
	Angiv \(T\)'s ækvivalens-klasser.\\ \\
	Ækvivalens-klasserne er:\\
	\(\{1, 3\}\) og \(\{2, 4\}\)\\
\end{enumerate}
	
	
	
	\section*{Opgave 4}
	

	
	
	
	
\end{document}