\documentclass[20pt]{article}
\usepackage{graphicx}
\usepackage{tabularx}
\usepackage[utf8]{inputenc}
\usepackage{fancyhdr}
\usepackage{amsmath}
\usepackage[danish]{babel}

\title{Take-Home Eksamen DM500 Studieintroduktion til datalogi, Efteråret 2021}
\author{Danny Nicolai Larsen (dalar21), Steffen Bach (stbac21),\\ Mikkel Brix Nielsen (mikke21) \\ Studiegruppe 11}
\date{\today}
\usepackage[utf8]{inputenc}
\pagestyle{fancy}
\renewcommand{\headrulewidth}{0pt}
\fancyhf{}
\chead{Git og Latex introduktion - DM500 Studieintroduktion}
\begin{document}
	\maketitle
	\newpage
	\section*{Opgave 1 - Eksamen januar 2009 opgave 1}
	Dem første opgave der vil blive løst kommer fra Eksamens sættet fra januar 2009, og er opgave nr.2.
	I denne opgaven skal vi betragte de to matricer:
	$$ 
	A = 
	\begin{bmatrix} 
		1 & 1 & 2 \\
		0 & 1 & 0 \\
		1 & 0 & 3 \\
	\end{bmatrix}
 	\quad og \quad B = 
	\begin{bmatrix} 
		0 & 2 & 1 \\
		0 & 0 & 0 \\
		2 & 0 & 1 \\
	\end{bmatrix}
	\quad
	$$
	Til denne opgaver forekommer 2 delopgaver. I den ene delopgave skal matricen der forekommer ved \(A+B\) beregnes, og i den anden delopgave skal matricen, der forekommer ved \(A*B\) beregnes.	
	
	\subsection*{Opaver a)}
	Lad matricen der forekommer ved \(A+B\) være C, så kan værdien, der skal indsættes i  matrice C, på index (i, j), være summen af tallene, der findes i matrice A på index (i, j) samt værdien, der findes i matrice B på index (i, j).
	
	\begin{equation}
		C = A+B
	\end{equation}

	\begin{equation}
			C = 
		\begin{bmatrix} 
			1 & 1 & 2 \\
			0 & 1 & 0 \\
			1 & 0 & 3 \\
		\end{bmatrix}
		\quad + \quad
		\begin{bmatrix} 
			0 & 2 & 1 \\
			0 & 0 & 0 \\
			2 & 0 & 1 \\
		\end{bmatrix}
		\quad	
	\end{equation}
	\\ 
	\begin{equation}
		= 
		\begin{bmatrix} 
			1+0 & 1+2 & 2+1 \\
			0+0 & 1+0 & 0+0 \\
			1+1 & 0+0 & 3+1 \\
		\end{bmatrix}
		= 
		\begin{bmatrix} 
			1 & 3 & 3 \\
			0 & 1 & 0 \\
			2 & 0 & 4 \\
		\end{bmatrix}
	\end{equation}
	\\
	\begin{equation}
		C = 
		\begin{bmatrix} 
			1 & 3 & 3 \\
			0 & 1 & 0 \\
			2 & 0 & 4 \\
		\end{bmatrix}
	\end{equation}
	\\
	Og herved er delopgave a løst, og den matrice der forekommer ved at addere matrice A med matrice B, \(A+B\), vil svare til matrice C.
	\newpage
	\subsection*{Opgave b)}
	Lad matricen der forekommer ved \(A*B\) være C, så kan værdien, der skal indsættes i matricen C på index (i, j) være summen af produkterne, der optår, når index (i, j) fra matrice A multipliceres med index (j, i) fra matrice B, for alle elementer på række i fra matrice A og kolonne j fra matrice B.
		\begin{equation}
		C = A*B
	\end{equation}
	
	\begin{equation}
		C = 
		\begin{bmatrix} 
			1 & 1 & 2 \\
			0 & 1 & 0 \\
			1 & 0 & 3 \\
		\end{bmatrix}
		\quad * \quad
		\begin{bmatrix} 
			0 & 2 & 1 \\
			0 & 0 & 0 \\
			2 & 0 & 1 \\
		\end{bmatrix}
		\quad	
	\end{equation}
	\\ 
	\begin{equation}
		= 
		\begin{bmatrix} 
			1*0+1*0+2*2 & 1*2+1*0+2*0 & 1*1+1*0+2*1 \\
			0*0+1*0+0*2 & 0*2+1*0+0*0 & 0*1+1*0+0*1 \\
			1*0+0*0+3*2 & 1*2+0*0+3*0 & 1*1+0*0+3*1 \\
		\end{bmatrix}
		= 
		\begin{bmatrix} 
			4 & 2 & 3 \\
			0 & 0 & 0 \\
			6 & 2 & 4 \\
		\end{bmatrix}
	\end{equation}
	\\
	\begin{equation}
		C = 
		\begin{bmatrix} 
			4 & 2 & 3 \\
			0 & 0 & 0 \\
			6 & 2 & 4 \\
		\end{bmatrix}
	\end{equation}
	\\
	Og herved er delopgave b løst, og den matrice der forekommer ved at multiplicere matrice A med matrice B, \(A*B\), vil svare til matrice C.
		
	\section*{Opgave 2}
	
	
	
	\section*{Opgave 3}
	
	
	
	\section*{Opgave 4}
	

	
	
	
	
\end{document}
